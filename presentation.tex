\documentclass{beamer}

\mode<presentation> {

  % The Beamer class comes with a number of default slide themes which
  % change the colors and layouts of slides. Below this is a list of
  % all the themes, uncomment each in turn to see what they look like.

%\usetheme{default}
%\usetheme{AnnArbor}
%\usetheme{Antibes}
%\usetheme{Bergen}
%\usetheme{Berkeley}
%\usetheme{Berlin}
%\usetheme{Boadilla}
%\usetheme{CambridgeUS}
%\usetheme{Copenhagen}
%\usetheme{Darmstadt}
%\usetheme{Dresden}
%\usetheme{Frankfurt}
%\usetheme{Goettingen}
%\usetheme{Hannover}
%\usetheme{Ilmenau}
%\usetheme{JuanLesPins}
%\usetheme{Luebeck}
\usetheme{Madrid}
%\usetheme{Malmoe}
%\usetheme{Marburg}
%\usetheme{Montpellier}
%\usetheme{PaloAlto}
%\usetheme{Pittsburgh}
%\usetheme{Rochester}
%\usetheme{Singapore}
%\usetheme{Szeged}
%\usetheme{Warsaw}

% As well as themes, the Beamer class has a number of color themes for
% any slide theme. Uncomment each of these in turn to see how it
% changes the colors of your current slide theme.

%\usecolortheme{albatross}
%\usecolortheme{beaver}
%\usecolortheme{beetle}
%\usecolortheme{crane}
%\usecolortheme{dolphin}
%\usecolortheme{dove}
%\usecolortheme{fly}
%\usecolortheme{lily}
%\usecolortheme{orchid}
%\usecolortheme{rose}
%\usecolortheme{seagull}
%\usecolortheme{seahorse}
%\usecolortheme{whale}
%\usecolortheme{wolverine}

%\setbeamertemplate{footline} % To remove the footer line in all slides uncomment this line
%\setbeamertemplate{footline}[page number] % To replace the footer line in all slides with a simple slide count uncomment this line

\setbeamertemplate{navigation symbols}{} % To remove the navigation symbols from the bottom of all slides uncomment this line
}

\usepackage{graphicx} % Allows including images
\usepackage[utf8]{inputenc}
\usepackage[T1]{fontenc}
\usepackage{tikz}

\title[RenderGirl]{\emph{RenderGirl} - A modular GPU raytracer using OpenCL for non-interactive graphics}

\author{Henrique Jung}
\institute[Unisinos]
{
Universidade do Vale do Rio dos Sinos \\
\medskip
\textit{henriquenj@gmail.com}
}
\date{\today}

\begin{document}

\begin{frame}
\titlepage % Print the title page as the first slide
\end{frame}

\begin{frame}
\frametitle{Overview}
\tableofcontents
\end{frame}

\section{Introduction}

\subsection{Motivation}
\begin{frame}
\frametitle{Introduction}

\begin{itemize}
\item Motivation

\begin{itemize}
\item Study the applications of GPU and GPU programming for non
  real-time graphics.
\item Offer more options for developers and users intersted in using
  dedicated hardware to perform their rendering.
\end{itemize}

\end{itemize}
\end{frame}


\subsection{Objectives}
\begin{frame}
\frametitle{Introduction}

\begin{itemize}
\item Objectives
\begin{itemize}
\item Create a raytracer capable of using specialized hardware while
  maintaining vendor agnosticism through OpenCL.
\item Design and implement a modular architecture with known
  interfaces capable of being plugged into a host 3D program.
\item Implement and study the applications of accelaration strucutres
  in the context of raytracing on many-core architectures, with focus
  on GPU.
\end{itemize}
\end{itemize}

\end{frame}


\section{Theory}
\begin{frame}
\frametitle{Theory}

\begin{itemize}

\item \emph{RenderGirl} is a raytracer renderer designed to run within the
OpenCL framework, which can take advantage of \emph{heteregenous
  platforms} consisting of GPUs, CPUs, FPGAs and some other types of
hardware.

\item Raytracing is a rendering algorithm based on the concept of
\emph{rays}, which are sent from a given point in space, creating the
concept of a camera. Each ray is tested against collision with the
objects in the scene, and if a collision is found, we may paint that
pixel with the color of the object.

\item Usually too computationally expensive to be used on real-time
  graphics.

\item Usually implemented with an accelaration structure that can
  apply a \emph{heuristic} in order optimize the rendering process.

\end{itemize}

\end{frame}



\section{Related work}
\begin{frame}
\frametitle{Related work}

Recent work related with GPU raytracing:

\begin{itemize}

\item Applications such as Blender Cycles, Octane render, V-ray
  renderer, Redshift and Indigo Renderer. Indigo and Cycles both claim
  to run using the OpenCL frmework.

\item OptiX from NVIDIA, which aims to provide a general purpose
  raytracing engine that can run on the CUDA platform.

\item Áfra and Szirmay-Kalos propose a novel traversal algorithm for
  BVH that doesn't use a stack\cite{Afra}.

\item Kalajanov and colleagues propose an acceleration structure using
  hierarchical two-level grids implemented in CUDA\cite{Kalojanov}.

\end{itemize}

\end{frame}



\begin{frame}
\frametitle{Related work}

\begin{itemize}

\item Ravichandran and colleagues propose a parallel divide and
  conquer raytracing suited for GPUs\cite{Ravichandran}.

\item Shumskiy and Parshin write a comparative study of ray-triangle
  intersection algorithms\cite{Shumskiy}.

\item Wong and colleagues propose an optimization method for GPU ray
  tracing by dividing objects into view-sets based on light and camera
  position\cite{Wong}.

\item Widmer and colleagues propose efficient hybrid acceleration
  structure for real-time screen space raytracing\cite{Widmer}.

\end{itemize}

\end{frame}

\section{RenderGirl}
\begin{frame}
\frametitle{RenderGirl}

\begin{itemize}

\item RenderGirl is composed of a \emph{core} that communicates with a
  given \emph{interface}.

\item The core portion is a static library that provides an API for
  receiving the 3D scene structure - vertices, triangles, objects,
  cameras and lights. It performs all of the communication between
  OpenCL and the interfaces, handling its context. It outputs an array
  of pixels with the rendered frame.

\item Interfaces link with the core at compile time and provide
  communication with a given host program.

\end{itemize}

\end{frame}

\begin{frame}
  \frametitle{RenderGirl}

Features supported:

\begin{itemize}
\item Rendering of triangulated scenes composed of several objects
\item Flat shading
\item OBJ-like materials (except on Blender plugin)
\item FXAA implemented in OpenCL
\item BVH partioning structure that optimizes the rendering process.
\end{itemize}

\end{frame}

\subsection{RenderGirl architecture}
\begin{frame}
\frametitle{RenderGirl architecture}

\begin{figure}
\centering
\usetikzlibrary{positioning,calc,fit}

\definecolor{black_red}{RGB}{179,0,0}
\definecolor{black}{RGB}{0,0,0}
\definecolor{blue}{RGB}{0,51,153}
\definecolor{orange}{RGB}{255,128,0}


\pgfdeclarelayer{core}
\pgfdeclarelayer{runtime}
\pgfsetlayers{core,main}

\pgfkeys{
  /tikz/node distance/.append code={
    \pgfkeyssetvalue{/tikz/node distance value}{#1}
  }
}

\newlength\myframesep
\setlength\myframesep{0pt}

\newcommand\widernode[5][core_node]{
\node[
        #1,
        inner sep=0pt,
        shift=($(#2.south)-(#2.north)$),
        yshift=-\pgfkeysvalueof{/tikz/node distance value},
        fit={(#2) (#3)},
        label=center:{\color{white}#4}] (#5) {};
}
% fix widernodeabove function, right now will shift nodes on X axis
\newcommand\widernodeabove[5][core_node]{
\node[
        #1,
        inner sep=0pt,
        shift=($(#2.north)+(#2.north)$),
        yshift=+\pgfkeysvalueof{/tikz/node distance value},
        fit={(#2) (#3)},
        label=center:{\color{white}#4}] (#5) {};
}

\begin{tikzpicture}[node distance=3pt,outer sep=0pt,
core_node/.style={
  draw=black,
  fill=black_red,
  rounded corners,
  font={\color{white}},
  align=center,
  minimum width = 70pt,
  text height=12pt,
  text depth=6pt},
other_node/.style={
  fill=black,
  draw=black,
  rounded corners,
  font={\color{white}},
  align=center,
  minimum width = 60pt,
  text height=12pt,
  text depth=6pt},
interface_node/.style={
  fill=orange,
  draw=black,
  rounded corners,
  font={\color{white}},
  align=center,
  minimum width = 70pt,
  text height=12pt,
  text depth=6pt},
host_node/.style={
  fill=black,
  draw=black,
  rounded corners,
  font={\color{white}},
  align=center,
  minimum width = 216pt,
  text height=12pt,
  text depth=6pt},
]

% core nodes

\node[core_node] (log) {Log subsystem};
\node[core_node, right=of log] (scene) {Scene Manager};
\node[core_node, right=of scene] (shared) {RenderGirl shared};
\widernode{log}{scene}{OpenCl wrapper}{wrapper};
\node[core_node, below=of shared] (kernels) {Kernels};

% interface

\node[interface_node, above=of scene](wrapper_host){Host structures};
\node[interface_node, above=of log](listeners){Log Listeners};
\node[interface_node, above=of shared](flow){Flow control};

\begin{pgfonlayer}{core}
\draw[core_node,draw=black,fill=black_red!40]
 ([xshift=-\myframesep,yshift=3\myframesep]current bounding box.north west)
    rectangle
  ([xshift=\myframesep,yshift=-\myframesep]current bounding box.south east);
\end{pgfonlayer}

% opencl runtime nodes

\widernode[other_node]{wrapper}{wrapper}{OpenCL runtime driver}{runtime}
\node[other_node, below=of kernels](compiler){OpenCL Compiler};

%  hardware

\widernode[other_node]{runtime}{compiler}{Hardware}{hardware}

% host 3d software

\node[host_node, above=of wrapper_host](host){Host 3D software};

\end{tikzpicture}
\caption{Architecture of RenderGirl. Red represents core components,
  orange represents interface components.}
\label{fig:architecture}
\end{figure}

\end{frame}


\begin{frame}
\frametitle{RenderGirl architecture}
\begin{itemize}
\item It was developed three interfaces: console interface, wxWidgets
  interface and a Blender Plugin interface.
\item The focus was to make an unmodified core works on all these
  interfaces.
\item Challanges with Blender plugin due to only use Python API.
\end{itemize}
\end{frame}



\subsection{Blender plugin}
\begin{frame}
\frametitle{Blender plugin}
\begin{itemize}
\item Blender only accepts python as scripting language for developing
  3rd party renderers.
\item It was necessary to connect the C++ from the core with a Python module loaded by Blender.
\item Several solutions proposed: Swig, Python C API.
\item In the end it was used the \emph{ctypes} python module, capable
  of loading shared libraries if there is an C API.
\end{itemize}
\end{frame}

% TODO: add blender plugin architecture drawning



\begin{frame}[allowframebreaks]
\frametitle{References}
\bibliographystyle{amsalpha}    %TODO: fix references markers on
                                %slides
\bibliography{references}
\end{frame}

\end{document}