% Uses a slightly modified IEEE VGTC template in conference mode
\documentclass{vgtc}
\usepackage{times}
\usepackage{graphicx}

\usepackage[utf8]{inputenc}
\usepackage[T1]{fontenc}
\usepackage{tikz}
\usepackage{algorithm}
\usepackage{algpseudocode}
\usepackage{pifont}


% If submitting a paper to a conference for review with a double
% blind reviewing process, replace the value ``0'' below with your
% OnlineID. Otherwise, leave it at ``0''.
\onlineid{0}

\vgtccategory{Research}

\usepackage{url}

%% Paper title.
\title{A modular GPU raytracer using OpenCL for non-interactive graphics}

\author{Henrique Nunes Jung\thanks{e-mail: henriquenj@gmail.com}
\and Vinicius Jurinic Cassol\thanks{e-mail: vjcassol@unisinos.br}}
\affiliation{\scriptsize Universidade do Vale do Rio dos Sinos}

% TODO: add a teaser image later on
% \teaser{
%   \includegraphics[width=2in]{paulista1891.jpg}
%   \caption{Lookit! Lookit!}
% }


%% Abstract section.

\abstract{

  We describe the development of a modular plug-in based
  raytracer renderer called \emph{RenderGirl} suitable for running
  inside the OpenCL framework. We aim to take advantage of
  heterogeneous computing devices such as GPUs and many-core CPUs,
  focusing on parallelism. We implemented the traditional partitioning
  scheme called \emph{bounding volume hierarchies}, where each scene
  is hierarchically subdivided into axis-aligned bounding boxes, so a
  ray may only need to traversal a subset of geometry by traversing
  the BVH and discarding objects it surely cannot hit, optimizing the
  rendering process. These structures were implemented on a many-core
  high parallel architecture suitable for OpenCL, which needed
  specific binary tree structure implementation ready for stackless
  traversal on GPUs. RenderGirl is split between two main portions:
  Core and Interface, where Core portions provides the bulk of
  ray-tracing operations and manages the communication with OpenCL;
  and interfaces that provides communication with a given host
  program, seeking modularity. In this paper we describe our results
  and performance gains with our partitioning scheme.

\smallskip

\noindent \textbf{Keywords:} GPGPU, OpenCL, computer graphics, raytracing
} % end abstract

\begin{document}

\firstsection{Introduction}

\maketitle

Platforms for general purpose computing using GPUs started when
NVIDIA\footnote{NVIDIA website \url{http://www.nvidia.com/}} released
its GPGPU platform called CUDA\footnote{CUDA homepage
 \url{http://www.nvidia.com/object/cuda_home_new.html}}, which is
capable of running only on NVIDIA hardware. OpenCL\footnote{OpenCL on
 Khronos Group website \url{https://www.khronos.org/opencl/}} was
born sometime later using the label \emph{heterogeneous computing}, in
order to specifies that the platform was supported by many kinds of
hardware and computers processors, including CPUs and non-NVIDIA GPUs,
which could be implemented by any vendor interested in developing
hardware focused on parallel computation.

Within this context, this work aims to provide a modular plugin-based
raytracer software that takes advantage of the parallel capabilities
of modern GPUs while maintaining hardware agnosticism by using the
portability OpenCL provides to hardware vendors and developers. By
being modular, it's also capable of running inside larger 3D suites
such as Blender, this way we can delegate other tasks to the host 3D
program and focuses on the rendering tasks.
% TODO: add here the method of BVH once it's selected

We propose a C++ application with wrappers for
Python %TODO: do I really need a python wrapper?
and C++ and a reference plugin for Blender, used to validate the
plugin architecture. The code has only portable system calls and
OpenCL function calls. In order to guarantee compilation on multiples
compilers, CMake was used as build system. Development happens on
GitHub, which contains the code, issue tracker and
documentation.

Intrinsic problems that arrives with GPU ray tracing includes the cost
of synchronizing and stopping working threads, poor branch prediction,
high latency of different OpenCL memory models, high latency on the
bus used between GPU and main memory (usually PCI Express) and the
broad variety of GPUs and OpenCL capable hardware, which can generate
different results on different setups.
% TODO add benefits of programming using GPUs, start with "by contrast"
% TODO find source for this

There are several acceleration structures for increase the efficiency
of rendering, which are called \emph{spatial data structures}. Some of
these structures are \emph{bounding volume hierarchies} (BVH),
\emph{binary space partitioning}, trees, quad-trees and octrees. They
hierarchically subdivided a scene so the queries for the objects become
faster \cite[Chapter~14.1]{akenine-moller:2008}. Non real-time renderers
do not have these constraints and therefore can employ expensive
algorithms, generating images of higher quality, while taking
advantage of the spatial structures in order to speed up
operations. This is the case for our work.

Our contributions include:

\begin{itemize}
  \item A renderer independent of a given 3D software.
  \item A raytracer designed to take advantage of parallel
    computation capabilities of modern GPUs and other OpenCL-capable
    devices.
  \item A modular architecture with replaceable components that
    communicates with a known interface.
  \item Usage of acceleration structures suitable for storing scene
    information inside OpenCL architecture.

\end{itemize}

\section{Related Work}
\label{sec:related-work}

Applications that provides GPU rendering capabilities for
non-interactive graphics include Blender Cycles\footnote{Cycles
  website \url{http://www.blender.org/manual/render/cycles/}} , Octane
render \footnote{Octane website
  \url{https://home.otoy.com/render/octane-render/}}, V-Ray renderer
\footnote{VRAY website
  \url{http://www.chaosgroup.com/en/2/vray.html}}, Redshift
\footnote{Redshift website
  \url{https://www.redshift3d.com/products/redshift}} and Indigo
renderer \footnote{Indigo website
  \url{http://www.indigorenderer.com/}}. Indigo and Cycles claims to
run using the OpenCL framework. Cycles is the only one that is free
software; originally it only supported CUDA-capable devices, but newer
versions of Blender ship with Cycles that works with limited features
also on OpenCL, although the developers specifies that CUDA platform
is more mature \footnote{Cycles GPU rendering page
  \url{http://www.blender.org/manual/render/cycles/gpu_rendering.html}}.
Also Cycles takes a different approach for device abstraction by
performing \emph{platform abstraction} by issuing commands to a common
\emph{device interface} that can redirect function calls to CUDA,
OpenCL, CPU or network; although the latter is still experimental
\footnote{Blender Cycles Developer Documentation
  \url{http://wiki.blender.org/index.php/Dev:2.6/Source/Render/Cycles/Devices}}. Our
work delegate the device abstraction entirely to OpenCL.

NVIDIA developed a framework called NVIDIA OptiX \footnote{OptiX
  website \url{https://developer.nvidia.com/optix}} which is intended
to be a general purpose ray tracing engine designed for NVIDIA GPUs
and other many-core architectures. It can be used for a variety of
tracing-based algorithms, and also on domains beyond computer graphics
such as sound propagation, collision-detection and artificial
intelligence. They aim to provide a set of programmable operations for
implementing ray tracing operations, in a pipeline model comparable
with OpenGL and Direct3D \cite{Parker:2010:OGP:1778765.1778803}.

Recent related work on the computing literature includes: Áfra and
Szirmay-Kalos propose a novel traversal algorithm for BVH that doesn't
uses a stack, and can execute both on CPU and GPU platforms. They
introduce the concept of \emph{bitstack}, which is an integer used in
the place of a stack \cite{Afra}. Kalajanov and colleagues propose an
acceleration structure using hierarchical two-level grids implemented
in CUDA, which can eliminated problems arriving from using a single
uniform grid for subdividing a scene. They call "teapot in a stadium
problem", where a great amount of objects is allocated on a single
cell of the uniform grid \cite{Kalojanov}. Ravichandran and colleagues
proposes a parallel divide and conquer raytracing suited for
GPUs. Divide and conquer ray tracing remove the need of creating an
explicit acceleration structures once per frame, instead it creates an
approximation using bound boxes \cite{Ravichandran}. Shumskiy and
Parshin write a comparative study of ray-triangle intersection
algorithms, how they perform on GPU hardware and how BVH acceleration
structures can be optimized for each one. They point out that some of
their results differs from generation to generation of the same line
of GPUS due to change in the micro-architecture.
% TODO: this last sentece should go to other section, more specific
\cite{Shumskiy}. Wong and colleagues propose an optimization method for
GPU ray tracing by dividing objects into view-sets based on light and
camera position, the BVH is then built using these views-sets, this
way the amount of triangles on the BVH is reduced \cite{Wong}. Widmer
and colleagues proposes efficient acceleration structure for real-time
screen space raytracing, they use an hybrid data structure combining
BVH and local planar approximation \cite{Widmer}. A great portion of
these works deal with acceleration structures for GPU raytracing -
e.g. kd-trees, BVHs -, which indicates the complexity of these
structures on GPU hardware.


\subsection{Contextualization}

Our own implementation relies on ray-triangle intersection algorithm
developed by Möller and Trumbore that aims to be fast by avoiding
ray-plane intersections\cite{moller}.

While the software is intended to work on all hardware capable of
running OpenCL, we tried to focus on GPUs, so optimization for this
type of processor received preference. It's also worth noting that
this work does not have real-time constraints and it is designed to be
used as an \emph{offline} raytracing, so image quality is more
important than rendering times.

Our work differs from the previously mentioned publications by
focusing on modularity and portability, both of hardware and operating
system.

\section{Methodology}
\label{sec:methodology}

We developed a GPU raytracer using the OpenCL framework called
\emph{RenderGirl}\footnote{RenderGirl project page
\url{https://github.com/henriquenj/rendergirl}} licensed under
LGPL. Our work aims to be useful to architects interested in using
their GPU hardware for rendering models. We try not to pose any
restriction on the use of the software regarding its host program, so
anyone can implement their own interfaces.

Most part of the code is written using C++, with small portions
written in Python and OpenCL C. Tools used in the development include
Visual Studio for project files and compilation under Windows, Git for
source control.

The host program of RenderGirl is Blender, which provides all the base
functionality of a 3D software suite. RenderGirl connects with Blender
through its Python API and implements a \emph{RenderEngine}
interface. The host software performs all communication with the final
user through its own GUI.

The software can also run as an standalone executable file through its
command-line interface. This interface has minimum dependencies and
does not output any file, just reporting if the rendering was successful.

We don't try to implement features that are not intrinsic related with
rendering, so RenderGirl delegates other tasks to the host program as
much as possible. This include loading of 3D models, user interface
interactions, image presentation and image saving. We believe this is
most useful and optimized way to solve the problem at hand.

\section{Architecture}
\label{sec:architecture}

RenderGirl is composed of a \emph{core} that communicates with a given
\emph{interface}. The overall architecture is depicted at figure~\ref{fig:architecture}. Each of these components have a specific set
of tasks inside the program.

The core portion is a static library that provides an API for
receiving the 3D scene structure - vertices, triangles, objects,
cameras and lights - and OpenCL device selection. It performs all the
communication between OpenCL and the interfaces, handling its
context. The output of render function is an array of pixels with the
rendered frame. It also provides the Log Subsystem dedicated to log
operations. Most of the interactions with the raytracer occurs using a
shared object called \emph{RenderGirlShared}, which is implemented as
a shared singleton. Scene structures are managed by a singleton called
\emph{SceneManager}, providing an API for setting up a scene using
RenderGirl data structures; it's also responsible by subdividing the
scene into BVH structures and make it suitable for OpenCL memory
models.


The interfaces can be anything from executables to dynamic libraries,
depending on the plugin interface provided by the host 3D
program. Interfaces link with the core at compile time and translate
all the requests from host program. They can pick individual features
to support, and a given interface may support only a subset of
features of the core. Three interfaces are provided as examples, they
are:

\begin{itemize}
\item \emph{Console interface}: The simplest interface. It links with
  the core and compiles as an executable. It only provides options for
  OBJ loader and does not output any image file. It's useful for
  porting the Core for other platforms and make sure it's working.
\item \emph{wxWidgets interface}: compiles as an standalone
  application with a GUI for loading OBJ files and render models,
  using the wxWidgets\footnote{http://wxwidgets.org/} toolkit. The
  user can choose attributes of the scene like camera position, color
  of light source and the output image format. It's likely to be
  deprecated over the Blender plugin interface due to the amount of
  dependencies it carries.
\item \emph{Blender Plugin}: the Blender plugin is the interface we
  provide as reference design to other interfaces for RenderGirl.
\end{itemize}

The core API is designed to only have synchronous calls, so each
function call will block the program - including the rendering
function which can takes several minutes to complete. Every host
program will provided different asynchronous APIs for its plugin
system in a different manner (or not provide it at all) so we don't
want to impose an asynchronous model at this stage of development,
leaving it for the interfaces to implement their own. Since
concurrency model on the rendering portion is handled by OpenCL, the
only feature we lost by not exposing asynchronous APIs in the core is
real-time preview. As a initial design decision to ease portability,
the core contains no system-dependent calls apart from OpenCL itself.

\begin{figure}
\centering
\usetikzlibrary{positioning,calc,fit}

\definecolor{black_red}{RGB}{179,0,0}
\definecolor{black}{RGB}{0,0,0}
\definecolor{blue}{RGB}{0,51,153}
\definecolor{orange}{RGB}{255,128,0}


\pgfdeclarelayer{core}
\pgfdeclarelayer{runtime}
\pgfsetlayers{core,main}

\pgfkeys{
  /tikz/node distance/.append code={
    \pgfkeyssetvalue{/tikz/node distance value}{#1}
  }
}

\newlength\myframesep
\setlength\myframesep{0pt}

\newcommand\widernode[5][core_node]{
\node[
        #1,
        inner sep=0pt,
        shift=($(#2.south)-(#2.north)$),
        yshift=-\pgfkeysvalueof{/tikz/node distance value},
        fit={(#2) (#3)},
        label=center:{\color{white}#4}] (#5) {};
}
% fix widernodeabove function, right now will shift nodes on X axis
\newcommand\widernodeabove[5][core_node]{
\node[
        #1,
        inner sep=0pt,
        shift=($(#2.north)+(#2.north)$),
        yshift=+\pgfkeysvalueof{/tikz/node distance value},
        fit={(#2) (#3)},
        label=center:{\color{white}#4}] (#5) {};
}

\begin{tikzpicture}[node distance=3pt,outer sep=0pt,
core_node/.style={
  draw=black,
  fill=black_red,
  rounded corners,
  font={\color{white}},
  align=center,
  minimum width = 70pt,
  text height=12pt,
  text depth=6pt},
other_node/.style={
  fill=black,
  draw=black,
  rounded corners,
  font={\color{white}},
  align=center,
  minimum width = 60pt,
  text height=12pt,
  text depth=6pt},
interface_node/.style={
  fill=orange,
  draw=black,
  rounded corners,
  font={\color{white}},
  align=center,
  minimum width = 70pt,
  text height=12pt,
  text depth=6pt},
host_node/.style={
  fill=black,
  draw=black,
  rounded corners,
  font={\color{white}},
  align=center,
  minimum width = 216pt,
  text height=12pt,
  text depth=6pt},
]

% core nodes

\node[core_node] (log) {Log subsystem};
\node[core_node, right=of log] (scene) {Scene Manager};
\node[core_node, right=of scene] (shared) {RenderGirl shared};
\widernode{log}{scene}{OpenCl wrapper}{wrapper};
\node[core_node, below=of shared] (kernels) {Kernels};

% interface

\node[interface_node, above=of scene](wrapper_host){Host structures};
\node[interface_node, above=of log](listeners){Log Listeners};
\node[interface_node, above=of shared](flow){Flow control};

\begin{pgfonlayer}{core}
\draw[core_node,draw=black,fill=black_red!40]
 ([xshift=-\myframesep,yshift=3\myframesep]current bounding box.north west)
    rectangle
  ([xshift=\myframesep,yshift=-\myframesep]current bounding box.south east);
\end{pgfonlayer}

% opencl runtime nodes

\widernode[other_node]{wrapper}{wrapper}{OpenCL runtime driver}{runtime}
\node[other_node, below=of kernels](compiler){OpenCL Compiler};

%  hardware

\widernode[other_node]{runtime}{compiler}{Hardware}{hardware}

% host 3d software

\node[host_node, above=of wrapper_host](host){Host 3D software};

\end{tikzpicture}
\caption{Architecture of RenderGirl. Red represents core components,
  orange represents interface components.}
\label{fig:architecture}
\end{figure}


\subsection{OpenCL scheduling model}

OpenCL dispatch each execution of the raytracer by running a piece of
code called \emph{kernel}. Each kernel runs once inside a
\emph{work-item} which belongs to a \emph{work-group}. The exact
nature of the execution will be hardware dependent - e.g. a work-item
can become thread -, we assume that each work-item can work
independently with no knowledge of other work-items, and therefore the
kernels do not perform any synchronization.

We launch one work-item executing one kernel per pixel of the
image. Each kernel will then build a ray based on pixel location
within the image to be tested for collision against all triangles of
the scene. This approach works well in parallel because does not
require any synchronization among work-items, and hence it's well
suited for GPUs. By contrast, it's also very inefficient because it
does not perform any heuristic and will brute-force test collision
with all triangles for every pixel on the image.

With BVH, we aim to reduce the amount of work per pixel and test
collision of rays with only a subset of triangles. The SceneManager is
responsible for organizing the objects of the scene into the BVH
structure prior the rendering, by analyzing bounding boxes and the
distribution of objects. We don' plan to build the BVH inside
OpenCL. The Algorithm~\ref{alg:rendering} provides a pseudo code
with this process.

\begin{algorithm}
\caption{Rendering execution}
\label{alg:rendering}
\begin{algorithmic}[1]
\Procedure{Host preparation}{}
\State Build BVH for the entire scene
\State Dispatch BVH structures to OpenCL device
\EndProcedure
\Procedure{OpenCL execution}{}
\For{each work-item running inside the OpenCL device}
\State Build ray $r$ from pixel coordinates
\For{each top-level BVH $b$}
\If{$r$ collide with $b$}
\Return $b$
\EndIf
\EndFor
\State Fetch triangle list $T$ from $b$
\For{each triangle $t$ \Pisymbol{psy}{206} $T$}
\If { $r$ collides with $t$}
\State {Fetch material of $t$}
\State {Paint pixel according to material}
\EndIf
\EndFor
\EndFor
\EndProcedure
\end{algorithmic}
\end{algorithm}

% \subsection{Blender Plugin}
% TODO: blender plugin general architecutre with swig process and
% python API

\subsection{Acceleration structures}

Acceleration structures are commonly used on ray tracing in order to
avoid expensive and inefficient brute-force ray-triangle intersection
tests for the entire scene. With these structures, using a given ray,
we can query for only a subset of triangles and perform the
intersection tests on them.

Thrane and Simonsen conduct a study comparing three different
acceleration structures and their implementation on GPUs: bounding
volume hierarchies, kd-trees and uniform grids. On their experiments,
they concluded that BVH outperform the other two except for a few
cases, citing the simpler nature of the traversal code with minimal
branching as the likely responsible for the good results
\cite{Thrane}. After careful consideration of their results, we choose
BVH for our own implementation.

The process of using aceleration structures is usually divided in two
steps: \emph{construction} and \emph{traversal}. The BVH was
originally proposed by Kay and Kajiya\cite{kay1986ray}, they describes
a CPU only implementation using convex hulls as bounding volumes for
each level of the tree.

For our BVH in GPU, the \emph{construction} portion is roughtly the
same as described by Kay and Kajiya. The BVH is built as a binary tree
where at each level objects are sorted by their absolute position
either in X or Y axis, the collection of objects is then split into
half, composing the two childs of that node. The axis used for
ordering are swapped at each level of the tree in a attempt to make a
fair distribuition of objects. This process of ordering and division
repeats until the collection of objects is reduced to one, making it a
leaf node, which then contains all the geometry of that particular
object. The major difference between our approach is that we use
\emph{axis-aligned bouding boxes}(AABB) as bouding volumes instead of
convex hulls, since the traversal code is less complex, although with
a much less tightly fit bouding bolume. Thrane and Simonsen mention
that there's a great penalty of running complicated code on the
GPU\cite{Thrane}. Every node holds an AABB fitting all the geometry of
its child nodes, so a ray must only test a collision against the AABB,
if the collision fails, the algorithm can discart that sub-tree and
resume processing on the next sister node. If the program reaches a
leaf node, then the ray must be tested against all the geometry of
that object within the leaf node. All the construction phase happens
on the CPU.

On the \emph{traversal} step, we implemented the traversal algorithm
described by Thrane and Simonsen in their master thesis. The main
problem to be solved here is overcome the lack of stack on GPU
hardware, since tree traversal algorithms are usually implemented
using recursion. The key difference from traditional CPU
implementations is that we build a fixed-order \emph{traversal array}
on the CPU and only iterate over it, so we never transmit the tree to
the OpenCL device and rather the iteration itself. The traversal array
already contains the nodes we must intersect in the corret order. The
BVH node on the traversal array have three properties: the AABB that
fits all that node downwards, the \emph{escape index} and a index that
points to an object in the global list of objects. The object index is
only valid on child nodes, and it's flagged as -1 when on a middle
node. The escape index represents the index where the traversal should
resume if the current node's AABB fail to intersect with the ray. This
can be visualized on figure XXX. The traversal is done in a top botton
left to right. The algorithm for traversal is detailed on Algorithm~
\ref{alg:bvh-traversal}.


\begin{algorithm}
\caption{BVH traversal on OpenCL device}
\label{alg:bvh-traversal}
\begin{algorithmic}[1]
\State Given a ray $r$
\State $tarray\gets$ The traversal array
\State $s\gets$ size of traversal array
\State $index\gets 0$
\While{$index < s$ }
\State $node\gets$ element from $tarray$ at $index$
\If{$r$ intersects with $node$'s AABB}
\If{$node$ is a leaf node}
\State Executes ray-triangle intersection with all triangles on this leaf node and return the nearest
\Else
\State $index\gets$ $index$ + 1
\EndIf
\Else
\State $index\gets$ $node$'s escape index
\EndIf
\EndWhile
\end{algorithmic}
\end{algorithm}

This implementation is a little bit different than Thrane and Simonsen
because they implemented the ray-traversal in several passes, where
the control goes back to CPU, so they keep intermediate renderer
states and hit records. This approach is probably due to memory
contraints at the time the implementation was done. Our rendering
happens by enqueeing a single OpenCL kernel only once, and the control
does not goes back to the CPU until the frame is formed.

\subsection{Log subsystem}

The log subsystem provides an API for programs interested in capturing
logs from the core. A simple log output such as \emph{stdout} and
\emph{stderr} is not enough because it does not provide the choice of
the output method, which is dependent on the host program. We provide
a solution using a combination of the observer and singleton patterns.

From the core perspective, the log is a singleton which can be called
at any point of the raytracer, with no knowledge of the destination of
the message. It provides two functions: \emph{message} and
\emph{error}. \emph{Message} can be used by any message treated as a normal
operation of the raytracer, such as information about state of
devices, OpenCL execution and rendering times. \emph{Error} is used for
logging errors, including OpenCL C syntax errors, which are generated at
run-time by the driver's compiler.

From the perspective of the interfaces, each of them need to implement
a \emph{LogListener} interface which provides the callbacks for
receiving the messages. Each listener object register with the log
subsystem, which will keep a list of listeners and will forward the
messages to all the objects on the list. The interface can then
capture the messages and print it to the user.

\subsection{Results}
\label{sec:results}


We tested our raytracer implementaton by running it from Blender using
a variaty of scene configurations. We aim to test different kinds of
object distributions in order to test how the BVH performs on each
case. We used the followng test setups:


\begin{itemize}
  \item A simples scene depicting a kitchen.
  \item The same scene as before but with several evenly distributed kitchens.
  \item The first scene but with different resolution sizes.
  \item A scene with several copies of the "Suzanne" model that comes
    with Blender. All models but one are outside the reach of the
    camera.                     %TODO: maybe not the best sentence?
\end{itemize}


\section{Conclusions}
\label{sec:conclusion}

With this work, we aim to offer more options for developers of 3D
modeling programs that want to incorporate a GPU raytracer into their
software without writing one from scratch. To the best of our
knowledge there's no free software available with the same design
goals.

\section{Future work}
\label{sec:future}


% \section*{Acknowledgments}


\bibliographystyle{abbrv}
\bibliography{references}
\end{document}
