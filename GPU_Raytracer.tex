\documentclass[a4paper]{sbgames}               % final
\usepackage{times}
\usepackage{graphicx}

%% use this for zero \parindent and non-zero \parskip, intelligently.
\usepackage{parskip}

%% the 'caption' package provides a nicer-looking replacement
\usepackage[labelfont=bf,textfont=it]{caption}

\usepackage{url}

%% Paper title.
\title{A GPU raytracer using OpenCL for non-interactive graphics}

%% Author and Affiliation (multiple authors). Use: and between authors

\author{Henrique Nunes Jung\\ Univesidade do Vale do Rio dos Sinos
}
\contactinfo{\{henriquenj\}@gmail.com
}
%% Keywords that describe your work.
\keywords{GPGPU, OpenCL, computer graphics, raytracing}

%% Start of the paper
% Attention: As you need to insert EPS images in Postscript,
% you need to insert PDF images into PDFs.
% In the text, extensions cancbe omitted (latex use .eps, pdflatex get .pdf)
% To convert them: epstopdf myimage.eps
\begin{document}

\teaser{

}

%% The ``\maketitle'' command must be the first command after the
%% ``\begin{document}'' command. It prepares and prints the title block.

\maketitle

%% Abstract section.

\begin{abstract}

  In this paper we describe the implementation of a modular raytracer
  renderer suitable for running inside the OpenCL architecture.
\end{abstract}

%% The ``\keywordlist'' command prints out the keywords.
\keywordlist
\contactlist

\section{Introduction}

\section{Related Work}
\label{sec:related-work}

Applications that provides GPU rendering capabilities for
non-interactive graphics include Blender Cycles\cite{BlenderCyclesDoc}
and Pixar RenderMan. % TODO: add reference here

% add two articles

\section{Conclusion}
\label{sec:conclusion}

\section*{Acknowledgements}


\bibliographystyle{sbgames}
\bibliography{references}
\end{document}
