%%%%%%%%%%%%%%%%%%%%%%%%%%%%%%%%%%%%%%%%%
% a0poster Portrait Poster
% LaTeX Template
% Version 1.0 (22/06/13)
%
% The a0poster class was created by:
% Gerlinde Kettl and Matthias Weiser (tex@kettl.de)
% Modified by Henrique Jung
%
% This template has been downloaded from:
% http://www.LaTeXTemplates.com
%
% License:
% CC BY-NC-SA 3.0 (http://creativecommons.org/licenses/by-nc-sa/3.0/)
%
%%%%%%%%%%%%%%%%%%%%%%%%%%%%%%%%%%%%%%%%%

\documentclass[a0,portrait]{a0poster}

\usepackage{multicol} % This is so we can have multiple columns of text side-by-side
\columnsep=100pt % This is the amount of white space between the columns in the poster
\columnseprule=3pt % This is the thickness of the black line between the columns in the poster

\usepackage[svgnames]{xcolor} % Specify colors by their 'svgnames',
                              % for a full list of all colors
                              % available see here:
                              % http://www.latextemplates.com/svgnames-colors

\usepackage{times} % Use the times font
% \usepackage{palatino} % Uncomment to use the Palatino font

\usepackage[scaled=2]{helvet}
\tiny

\usepackage{graphicx} % Required for including images
\graphicspath{{figures/}} % Location of the graphics files
\usepackage{booktabs} % Top and bottom rules for table
\usepackage[font=small,labelfont=bf]{caption} % Required for specifying captions to tables and figures
\usepackage{amsfonts, amsmath, amsthm, amssymb} % For math fonts, symbols and environments
\usepackage{wrapfig} % Allows wrapping text around tables and figures

\begin{document}

\begin{minipage}[b]{0.75\linewidth}
  \veryHuge \color{NavyBlue} \textbf{A modular GPU raytracer using
    OpenCL for non-interactive graphics} \color{Black}\\[2cm] % Title
  \huge \textbf{Henrique Nunes Jung \& Vinicius Jurinic
    Cassol}\\[0.5cm] % Author(s)
  \huge Universidade do Vale do Rio dos
  Sinos\\[0.4cm] % University/organization
  \Large \texttt{henriquenj@gmail.com} \\
  \Large \texttt{vjcassol@unisinos.br} \\
  \
\end{minipage}
%
\begin{minipage}[b]{0.25\linewidth}
\includegraphics[width=20cm]{logo_unisinos.png}\\
\end{minipage}

\vspace{1cm} % A bit of extra whitespace between the header and poster content

%----------------------------------------------------------------------------------------

\begin{multicols}{2} % This is how many columns your poster will be
                     % broken into, a portrait poster is generally
                     % split into 2 columns

%----------------------------------------------------------------------------------------
%	ABSTRACT
%----------------------------------------------------------------------------------------


\color{Navy} % Navy color for the abstract

\section*{Abstract}
\large

We describe the development of a modular plugin based raytracer
renderer called \emph{RenderGirl} suitable for running inside the
OpenCL framework. We aim to take advantage of heterogeneous computing
devices such as GPUs and many-core CPUs, focusing on parallelism. We
implemented the traditional partitioning scheme called \emph{bounding
  volume hierarchies}, where each scene is hierarchically subdivided
into axis-aligned bounding boxes, so a ray may only need to traverse a
subset of geometry by traversing the BVH and discarding objects it
surely cannot hit, optimizing the rendering process. These structures
were implemented on a many-core high parallel architecture suitable
for OpenCL, which needed a specific binary tree structure
implementation ready for stackless traversal on GPUs. RenderGirl is
split between two main portions: Core and Interface, where the Core
portions provide the bulk of ray-tracing operations and manage the
communication with OpenCL; and the interfaces provide communication
with a given host program, seeking modularity. In this work we
describe our results and performance gains with our partitioning
scheme.

%----------------------------------------------------------------------------------------
%	INTRODUCTION
%----------------------------------------------------------------------------------------

\color{SaddleBrown} % SaddleBrown color for the introduction
\large
\section*{Introduction}

Aliquam non lacus dolor, \textit{a aliquam quam}
\cite{Smith:2012qr}. Cum sociis natoque penatibus et magnis dis
parturient montes, nascetur ridiculus mus. Nulla in nibh mauris. Donec
vel ligula nisi, a lacinia arcu. Sed mi dui, malesuada vel consectetur
et, egestas porta nisi. Sed eleifend pharetra dolor, et dapibus est
vulputate eu. \textbf{Integer faucibus elementum felis vitae
  fringilla.} In hac habitasse platea dictumst. Duis tristique rutrum
nisl, nec vulputate elit porta ut. Donec sodales sollicitudin turpis
sed convallis. Etiam mauris ligula, blandit adipiscing condimentum eu,
dapibus pellentesque risus.

\textit{Aliquam auctor}, metus id ultrices porta, risus enim cursus
sapien, quis iaculis sapien tortor sed odio. Mauris ante orci, euismod
vitae tincidunt eu, porta ut neque. Aenean sapien est, viverra vel
lacinia nec, venenatis eu nulla. Maecenas ut nunc nibh, et tempus
libero. Aenean vitae risus ante. Pellentesque condimentum dui. Etiam
sagittis purus non tellus tempor volutpat. Donec et dui non massa
tristique adipiscing.

%----------------------------------------------------------------------------------------
%	OBJECTIVES
%----------------------------------------------------------------------------------------

\color{DarkSlateGray} % DarkSlateGray color for the rest of the content

\section*{Main Objectives}

\begin{enumerate}
\item Lorem ipsum dolor sit amet, consectetur.
\item Nullam at mi nisl. Vestibulum est purus, ultricies cursus volutpat sit amet, vestibulum eu.
\item Praesent tortor libero, vulputate quis elementum a, iaculis.
\item Phasellus a quam mauris, non varius mauris. Fusce tristique, enim tempor varius porta, elit purus commodo velit, pretium mattis ligula nisl nec ante.
\item Ut adipiscing accumsan sapien, sit amet pretium.
\item Estibulum est purus, ultricies cursus volutpat
\item Nullam at mi nisl. Vestibulum est purus, ultricies cursus volutpat sit amet, vestibulum eu.
\item Praesent tortor libero, vulputate quis elementum a, iaculis.
\end{enumerate}

%----------------------------------------------------------------------------------------
%	MATERIALS AND METHODS
%----------------------------------------------------------------------------------------

\section*{Materials and Methods}

Fusce magna risus, molestie ut porttitor in, consectetur sed
mi. Vestibulum ante ipsum primis in faucibus orci luctus et ultrices
posuere cubilia Curae; Pellentesque consectetur blandit
pellentesque. Sed odio justo, viverra nec porttitor vel, lacinia a
nunc. Suspendisse pulvinar euismod arcu, sit amet accumsan enim
fermentum quis. In id mauris ut dui feugiat egestas. Vestibulum ac
turpis lacinia nisl commodo sagittis eget sit amet sapien.


%----------------------------------------------------------------------------------------
%	RESULTS
%----------------------------------------------------------------------------------------

\section*{Results}


\begin{center}\vspace{1cm}
% \includegraphics[width=0.8\linewidth]{placeholder}
\captionof{figure}{\color{Green} Figure caption}
\end{center}\vspace{1cm}

In hac habitasse platea dictumst. Etiam placerat, risus ac.

Adipiscing lectus in magna blandit:

\begin{center}\vspace{1cm}
\begin{tabular}{l l l l}
\toprule
\textbf{Treatments} & \textbf{Response 1} & \textbf{Response 2} \\
\midrule
Treatment 1 & 0.0003262 & 0.562 \\
Treatment 2 & 0.0015681 & 0.910 \\
Treatment 3 & 0.0009271 & 0.296 \\
\bottomrule
\end{tabular}
\captionof{table}{\color{Green} Table caption}
\end{center}\vspace{1cm}

Vivamus sed nibh ac metus tristique tristique a vitae ante. Sed
lobortis mi ut arcu fringilla et adipiscing ligula rutrum. Aenean
turpis velit, placerat eget tincidunt nec, ornare in nisl. In
placerat.

\begin{center}\vspace{1cm}
% \includegraphics[width=0.8\linewidth]{placeholder}
\captionof{figure}{\color{Green} Figure caption}
\end{center}\vspace{1cm}

%----------------------------------------------------------------------------------------
%	CONCLUSIONS
%----------------------------------------------------------------------------------------

\color{SaddleBrown} % SaddleBrown color for the conclusions to make them stand out

\section*{Conclusions}

\begin{itemize}
\item Pellentesque eget orci eros. Fusce ultricies, tellus et pellentesque fringilla, ante massa luctus libero, quis tristique purus urna nec nibh. Phasellus fermentum rutrum elementum. Nam quis justo lectus.
\item Vestibulum sem ante, hendrerit a gravida ac, blandit quis magna.
\item Donec sem metus, facilisis at condimentum eget, vehicula ut massa. Morbi consequat, diam sed convallis tincidunt, arcu nunc.
\item Nunc at convallis urna. isus ante. Pellentesque condimentum dui. Etiam sagittis purus non tellus tempor volutpat. Donec et dui non massa tristique adipiscing.
\end{itemize}

\color{DarkSlateGray} % Set the color back to DarkSlateGray for the rest of the content

%----------------------------------------------------------------------------------------
%	FORTHCOMING RESEARCH
%----------------------------------------------------------------------------------------

\section*{Forthcoming Research}

Vivamus molestie, risus tempor vehicula mattis, libero arcu volutpat purus, sed blandit sem nibh eget turpis. Maecenas rutrum dui blandit lorem vulputate gravida. Praesent venenatis mi vel lorem tempor at varius diam sagittis. Nam eu leo id turpis interdum luctus a sed augue. Nam tellus.

 %----------------------------------------------------------------------------------------
%	REFERENCES
%----------------------------------------------------------------------------------------

\nocite{*} % Print all references regardless of whether they were cited in the poster or not
\bibliographystyle{plain} % Plain referencing style
\bibliography{sample} % Use the example bibliography file sample.bib

%----------------------------------------------------------------------------------------
%	ACKNOWLEDGEMENTS
%----------------------------------------------------------------------------------------

\section*{Acknowledgements}

Etiam fermentum, arcu ut gravida fringilla, dolor arcu laoreet justo, ut imperdiet urna arcu a arcu. Donec nec ante a dui tempus consectetur. Cras nisi turpis, dapibus sit amet mattis sed, laoreet.

%----------------------------------------------------------------------------------------

\end{multicols}
\end{document}